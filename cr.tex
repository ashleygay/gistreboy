\documentclass{article}

\usepackage[french]{babel}

\author{Corentin Gay\\Pierre Chavanne\\Pierre-Adrien Estanove\\Arnaud
Basti\'e}
\date{\today}

\title{Compte-rendu de r\'eunion de suivi de PFE du 14/12/2017}

\usepackage[a4paper,top=3cm,bottom=2cm,left=5cm,right=5cm,marginparwidth=1.75cm]{geometry}
\begin{document}
\maketitle

Version 1 du \today
\section{Pr\'esentation}
Le groupe GistreBoy est compos\'e des personnes suivantes:
\begin{itemize}
	\item Corentin Gay
	\item Arnaud Basti\'e
	\item Pierre-Adrien Estanove
	\item Pierre Chavanne
\end{itemize}


Les membres du groupe pr\'esents \'etaient:
\begin{itemize}
	\item Corentin Gay
	\item Arnaud Basti\'e
	\item Pierre-Adrien Estanove
	\item Pierre Chavanne
\end{itemize}


Le jury \'etait compos\'e de:
\begin{itemize}
	\item David Bouchet
\end{itemize}
\section{Points soulev\'es pendant la soutenance}

\subsection{Affichage}
Nous allons ajouter une option pour passer de la r\'esolution normale de la
Gameboy \`a un mode plein-\'ecran.

\subsection{Slides}
Nous utiliserons les termes francais. Par
exemple, nous remplacerons 'stack' par 'pile'. Nous pr\'eciserons si jamais
nous utilisons des termes anglais dans le cas d'un terme traduit qui serait
incomplet.

\subsection{Envoi de code}
Nous enverrons des exemples de code qui sont
repr\'esentatifs de notre projet.

\end{document}
