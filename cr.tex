\documentclass{article}

\usepackage[french]{babel}

\author{Corentin Gay\\Arnaud Basti\'e\\Pierre-Adrien Estanove\\Pierre Chavanne}
\date{\today}

\title{Compte-rendu de r\'eunion de suivi de PFE du 14/12/2017}

\usepackage[a4paper,top=3cm,bottom=2cm,left=5cm,right=5cm,marginparwidth=1.75cm]{geometry}
\begin{document}
\maketitle

Version 2 du \today
\section{Pr\'esentation}
Le lien du projet sur github est : https://github.com/corentingay/gistreboy
\\\\
Le groupe GistreBoy est compos\'e des personnes suivantes:
\begin{itemize}
	\item Corentin Gay
	\item Arnaud Basti\'e
	\item Pierre-Adrien Estanove
	\item Pierre Chavanne
\end{itemize}
Les membres du groupe pr\'esents \'etaient:
\begin{itemize}
	\item Corentin Gay
	\item Arnaud Basti\'e
	\item Pierre-Adrien Estanove
	\item Pierre Chavanne
\end{itemize}
Le jury \'etait compos\'e de:
\begin{itemize}
	\item David Bouchet
\end{itemize}
\section{Points soulev\'es pendant la soutenance}

\subsection{Affichage}
Nous allons ajouter une option pour passer de la r\'esolution normale de la
Gameboy \`a un mode plein-\'ecran.

\subsection{Slides}
Nous utiliserons les termes francais. Par
exemple, nous remplacerons 'stack' par 'pile'. Nous pr\'eciserons si jamais
nous utilisons des termes anglais dans le cas de mots intraduisibles.

\subsection{Envoi de code}
Nous enverrons des exemples de code qui sont repr\'esentatifs de notre projet.
Par exemple, pour la prochaine soutenance, nous enverrons les fichiers concernant:
\begin{itemize}
	\item le processeur
	\item la m\'emoire
	\item l'\'ecran LCD
	\item la classe Gameboy
	\item les instructions
\end{itemize}

\end{document}
